\chapter{Unidad II}
\section{Modelo OSI}
Pensemos en administración en la integración de varios elementos para manejar de forma eficiente y eficaz el manejar las edes de computadoras.

\begin{itemize}
    \item {Administrar: planificación de la mejor manera para una mejor gestión de la red-}
    \item {Gestionar: implementación de modificaciones y crreccionaes para alcanzar los objetivos de la red}
\end{itemize}

\subsubsection{Administración de sistemas}
Suma total de las políticas y procedimientos que intervienen en la configuración, control y monitoreo que conforman una red, con el fin de aseurar el eficiente y efectivo empleo de sus recursos. \\\\

Se le solicita a la ISO que diseñe un modelo de administración OSI 

\subsubsection{Modelo de administración OSI}

La Organización de Estándares INternacionales creó una comisión para crear un modelo de administración de redes, bajo la dirección del grupo OSI.

Surge como un modelo que involucra tanto la PC como la red, buscando una coordinación e integración entre si aún que se traten de modleos distintos.

\subsubsection{Modelo de administración de redes OSI (OSI-NMM)}
Es un modelo estándar que proporciona el marco conceptual para la organización de una amplia gama de recursos de la red.\\\\
Planificación de la capacidad de red\\\\
Gestión del rendimiento de la red\\

Comprende la administración de sistemas que delimita la operación de cualquiera de las 7 capas del modelo OSI, y la administración de los objetos gestionados,
Plantea los modelos de:
\begin{itemize}
    \item Organización
    \item Información
    \item Comunicación
    \item Función
\end{itemize}

\subsubsection{Modelo organizacional}
Describe lso componentes de la administración de redes tales como administrador, agente y otros, y sus interrelaciones.
Sus relaciones vienen dadas por la arquitectura de red.\\\\
El modelo organizacional del modelo OSI define los bloques y la relación entre estos.\\\\

Es una estructura dividida en dominios de red, los cuales comprenden su operabilidad y ofrece soporte de los aspectos de gestión del mismo.\\\\
Define conceptos para una gestión cooperativa, como para una gestión basada en jerarquías:
\begin{itemize}
    \item COncepto simétrico - entre dominios
    \item Concepto asimétrico - entre dominios y subdominios
\end{itemize}

\paragraph{Dominio ejemplo}
www.ipn.mx \\
\textbf{www.escom.ipn.mx}\\
\textbf{www.saes.escom.ipn.mx}
Dónde el tercer y segundo ejemplo son subdiminios del primero 
\subsubsection{Gestión de dominios}
Define la división de entorno, teniendo en cuenta dos moibos principales:
\begin{itemize}
    \item {Políticas funcionales, donde se incluyen políticas de seguridad contabilidad etc}
    \item {Políticas no funcionales, como la gestión geográfica, tecnologías, etc}
\end{itemize}
\subsubsection{Sub modelo informativo}
Trata de la estructura y almacenamiento de la información relativa a la administración de la red \\\\
Esta información se guarda en una base de datos la cual recube nombre de base de datos de información de administración \textbf{(MIB)} Es un archivo que guarda información sobre nuestras redes, el cual se encuentra en todas las capas de administración.
\subsubsection{Sub modelo conunicacional}
Habla de la forma como se comunican los datos de administración en el proceso gestir-agente\\\\
Atiende lo relacionado con el protocolo de transporte, el protocolo de aplicaciones y los comandos y respuestas entre pares. (Como me comunico, como lo hago, formato de mis tramas, que protocolos usaré, etc.)

\subsubsection{Modelo funcional}
Divide la complejidad de la administración en áreas funcionales de administración e intenta especificar funciones de administración genéricas.\\\\
EL modelo funcional proporciona las bases para construir librerías y soluciones.\\\\
áreas de administración del modelo OSI
\begin{itemize}
    \item {Administración de fallas (fault management)}
    \item {Administracion de configuración (configuration management)}
    \item {Administración de estadísticas y contabilidad (accounting amanager) [Comportamiento de la red y si todo está dentro de los rangos permitidos, además de saber cuanto se va a cobrar]}
    \item {Administración de desempeño (performance management) [Que todo este funcionando bien, saber si es posible hacer mejoras, etc.]}
    \item {Administración de seguridad (security management) [Quién tiene acceso a que si quien trata de accesar tiene permisos, etc]}
\end{itemize}

\subsection{CMIS (Servicios de interopabilidad de gestión de contenidos)}
Es un estándar, solo sabemos que cosas debe hacer. CMIS permite la interoperabilidad entre los distintos servicios que tenemos, el gestor hace pull-in y recaba información de todos los agentes.

\textbf{T-1.1 Levantar la siguiente topología GNS3 192.168.0.0/24: Bajar las ISOS router3600, Switch3600 crear dos máquinas virtuales basadas en linux. }
\begin{lstlisting}
    enable
    conf 
    configure ter 
    configure terminal 
    inte
    interface e 
    ip add 
    ip address
    ip address 192.268.0.1 255.255.255.128
    no shutdown
    e
    exit
    inte e 0/1
    ip address  
\end{lstlisting}

\section{CMIS (Servicio de interoperabilidad de gestión de contenidos)}
Surge como un impulso de varias instituciones privadas.\\\\

Estandar abierto de OASIS diseñado para la interoprabilidad de lso sistemas de gestión de contenidos a través d einternet\\

A través d euna capa de abstración permite la gestión de contenidos.\\\\

\subsection{Sistema de gestión de contenidos o CMS}
Podemos hacer una analogía con el periódico, antes si se compraba un periódico a cualquier hora del día tenías la misma información, hoy en día tienes información en tiempo real, además de tener aglomeradas varias fuentes. CMS funciona de esta manera, un manejo dinámico de la información.\\

Aplicación que explota un entorno de trabajo para la creación y administración de contenidos.

Se usa intensamente en páginas web.

Usa una o varias bases de datos para alojar el contenido del sitio web.

Permite manejar de forma independiente el contenido y el diseño.

\subsection{Objetivos}

CMIS fué diseñado para majorar los sitemas de administración de contenido empresarial que existen junto con sus interfaces de aplicación actual junto con una capa de abstracción que nos permite homogeneizar la información.

\subsection{Historia}
EMC, IBM y Microsoft propusieron este formato.

Las tres empresas enviaron de forma conjunta CMIS a OASIS. Esta propuesta fue aprobada en 2010 y tuvo una actualizació en 2012. 

\subsection{Definiciones}
\begin{itemize}
    \item Modelo de dominio
    \item Enlaces de servicios web
    \item Rest ful y AtomPub: Difusión web usando XML y un protocolo simple basado en HTTP
\end{itemize}

\subsection{ Enfoque de CMIS }
CMIS se enfoca en las capacidades de contenido básico d eun sistema de administración de contenido empresarial, las cuales son la \textbf{creación, lectura, escritura, borrar y funciones de petición} todo se maneja jerarquicamente como un árbol. 

\subsection{Alcances CMIS}
CMIS incorpora conceptos contemporáneos de una orientación de servicios web y especificaciones ambas basadas en SOAP y REST (Estos últimos nos permiten tener una interoperabilidad entre objetos). 

\subsection{CMIS}
Los trabajadores pueden usar una única aplicación para acceder e intercambiar contenido que esta almacenado en varios sistemas de gestión empresarial sin importar que sean diferentes las computadoras.

\subsubsection{Otros estándares}
\begin{itemize}
    \item JCR Java Content Repository (Bastante complicado de utilizar)
    \item WebDAV (Muy jóven)
\end{itemize}

\subsection{Funciones genéricas de CMIS}
Crear, acceder a versiones de documentos y crear, leer actualizar, relacionar y borrar obketos

\subsection{Object identity ODI}
En el momento que se crea un objeto se crea un OID el cual es intransfereible (como una primary key de un objeto).

\subsubsection{Jerarquía de documentos}
Se usan folders y usa unaestructura de árbol normal, se manejan archivos completos y sin flujo de contenidos
'\subsection{CRUD}
\begin{itemize}
    \item Creación
    \item Recuperación
    \item Actualización
    \item Eliminación
\end{itemize}

\section{CMIP}
CMIS es un protocolo!, no nos dice como hacer las cosas solo como deben hacerse las cosas.  Desarrollado por la ISO, ofrece un mecanismo de transporte en la forma de servicio pregunta-respuesta para las 7 capas del modelo OSI. Imaginemos que tenemos una PC que ejecuta un software de administración de servicios y diversos dispositivos de red (Switch, routers, PC, etc) en cada uno de esos dispositios está en ejecución un pequeño programa llamado agente, mientras que en en la PC que tiene el software de administración de servicios tenemos el gestor. El gestor usando pull-in hace preguntas a los agentes (Dame información en el MIB (Base de Información de Administración), el agente responde). El recaba información, realiza estadisticas etc. Posteriormente el gestor envia una respuesta con el fin de ajustar algo en los agentes. Finalmente los dispositivos de red responden si se ha hecho lo que el gestor ha pedido.\\\\

Es considerada como parte d euna arquitectura de administración de red, que provee mecanismos de intercambio de información, entre un administrador y elementos remotos de red, cuyo funcionamiento está basado en los servicios CMIP.\\\\

Los sistemas de administración de red, basados en CMIP son utilizados en la administración de:

\begin{itemize}
    \item Redes de área local, LAN.
    \item Redes corporativas y provadas de área amplia.
    \item Redes nacionales e internacionales.
\end{itemize}

Define la información en términos de objetos administrados.

Permite su modificación y acciones sobre archivos

\subsection{Sistemas de administración de red o NMS}
Es un conjunto de aplicaciones que supervisan y controlan los dispositivos administrados.

Proporcionan el volumen de recursos de procesamiento y memoria requeridos para la administración de la red.

Uno o más NMS deben existir en cualquier red.

\subsection{Unidad de datos de protocolo de aplicación}
Las tramas que utiliza CMIP para su funcionamiento

\subsection{Caracteristicas del protocolo}
\begin{itemize}
    \item Ocupa muchos recursos
    \item Tramas muy grandes
    \item Peticiones muy complicadas de realizar
\end{itemize}

\subsection{Protocolos del CMIP}
ACSE, ROSE, CMISE

\chapter{Configuración básica de R1 y Routers}
\section{Conexiones a la interfaz de comandos}
\begin{itemize}
    \item {Terminal de consola: podemos hacerlo de forma local}
    \item {Telnet/SSH: nuestro propia red de internet (La primera es sin protección la otra de forma segura)}
\end{itemize}
\section{Acceso a la interfaz de comando CLI.}
Interfaz de Línea de Comandos. Requerimos cuenta de usuario. La consola muestra algo más o menos así.

\begin{lstlisting}
    dispositivo:ruta>comando                \\Switch
    dispositivo:ruta#comando                \\Router
    dispositivo:ruta(enable)#comando      
    dispositivo:ruta(config)a#comando
    dispositivo:ruta(config-vlan)#comando
    dispositivo:ruta(config-it)#comando
\end{lstlisting}
Si en el prompt tenemos el símbolo $>$ estamos en modo switch y en modo router tenemos otro símbolo el cual provee más permisos, tenemos modo de privilegio y finalmente configuración específica.\\\\

Podemos tener del 0 - 15 modelos de dispositivo, en los dispositivos CISCO hay 0, 1 y 15. Podemos definir el medio y asignar diverss permisos o comandos de ejecucion generando nuevos grupos.

\subsection{Modos de acceso}
\begin{itemize}
    \item {Solo lectura: show}
    \item {Lectura-escritura: set show}
\end{itemize}

\subsection{USAMOS GNS3}
\begin{itemize}
    \item {1. crear proyecto nuevo}
    \item {2. Cuantas direcciones IP requiero para una comunicación troncal? Requiero dos, pero también la IP y dirección de broadcast, además la máscara es decir 10.10.0.0/30, }
\end{itemize}
Configuramos Router 1 abriendo consola:
\begin{lstlisting}
    R1#disable
    R1>enable 
    R1# ?               //Comandos a ejecutar
    R1# c?              //Comandos que inician con c
    R1# configure?      //'comandos' del comando
    R1# conf terminal          //configurar terminal 
    R1(config)#interface et
    R1(config)#interface ethernet 0/1
    R1(config-if)#ip a
    R1(config-if)#ip add
    R1(config-if)#ip address 192.168.0.1 255.255.255.0
    R1(config-if)#no shutdown
    R1(config-if)#exit 
    R1(config)# inter e0/0      //Segundo router
    R1(config-if)#ip add 
    R1(config-if)#ip address 10.10.0.0 255.255.255.252
    R1(config-if)#no shut 
    R1(config-if)#exit
    R1(config)#
\end{lstlisting}

Virtual PC
\begin{lstlisting}
    PC-1> ip ? // que podemos hacer 
    PC-1> ip address ip 192.168.0.2 24 192.168.0.1
    PC-1> ping 192.168.0.1
\end{lstlisting}

\begin{lstlisting}
    R1#show ip address brief
\end{lstlisting}

\begin{lstlisting}
    R1#show ip route    //Tabla de enrutamiento (Hasta a donde llega mi router)
\end{lstlisting}

Algunos comandos simples
\begin{lstlisting}
    R1#conf t
    R1(config)#prompt "Curso4CM1"
\end{lstlisting}
\begin{lstlisting}
    R1#conf t 
    R1(config)#host name "Curso4CM1"
    Curso4CM1#
\end{lstlisting}
Usando password
\begin{lstlisting}
    R1#conf t
    R1(config)#enable password 1234
    R1(config)#end 
    R1#disable 
    R1>enable //Ahora pedirá password
\end{lstlisting}

\begin{lstlisting}
    R1#show running-config //Configuración y comandos que debo ejecutar
\end{lstlisting}

\begin{lstlisting}
    R1#show running-config | include pass //Mostrar el password
\end{lstlisting}

Contraseña segura
\begin{lstlisting}
    R1#conf t
    R1(config)#enable secret 12345678
    R1#end 
    R1#show running-config | include enable
\end{lstlisting}

Activar el servicio de encriptación
\begin{lstlisting}
    R1#conf t
    R1(config)#service password-encryption
    R1(config)#exit
    R1#show running-config | include pass
\end{lstlisting}

Ya hemos crado un password, el cual tiene como fin el cambio de privilegios, pordemos también poner un password para abrir la consola.
\begin{lstlisting}
    R1#conf t
    R1(config)#line console 0
    R1(config-line)#password 1234
    R1(config-line)#login 
    R1(config-line)#exit 
    R1(config)#exit
    R1#disable 
    R1enable //Pedirá password 
\end{lstlisting}

Quiero conexión remota de R2 a R1\\
Puerto telnet = 22, requerimos un password.
\begin{lstlisting}
    R2#telnet 10.10.0.1 //Requiere password y no la tenemos, entonces debemos configurar primero a donde nos vamos a conectar (añadir password a R1)
\end{lstlisting}

Configuramos password para telnet
\begin{lstlisting}
    R1#conf t
    R1(config)#line vty 0 15 //Accesos completos, cisco usa permisos de seguridad de 0 a 15 pero solo tenemos acceso a 3 
    R1(config)#password 1234
\end{lstlisting}

Intentamos de nuevo con R2 conectarnos
\begin{lstlisting}
    R2#telnet 10.10.0.1 //Ahora nos pedirá password, para nuestro caso es 1234, ahora estamos en R1
    R1
\end{lstlisting}

Configuración de SSL: tiene mayor nivel de seguridad. Podemos encriptar las tramas
\begin{lstlisting}
    R1disable
    R1show ssh //Esta desactivado
\end{lstlisting}

Asignando llave y dominio 
\begin{lstlisting}
    R1enable
    R1#conf t
    R1(config)#ip domain-name curso4CM1.com//Configuramos el dominio, este tiene relación lógica o práctica algo que nos vuelve comunes.
    R1(config)#crypto key generate rsa general-keys //Definimos una llave de encriptación
    How many bits in the modulus [512]: 1024 //Asignamos el tamaño de llave en este caso 1024 bytes
\end{lstlisting}

Tiempo de inactividad, se cae la conexión en n segundos 
\begin{lstlisting}
    R1(config)#ip ssh time-out 30 //En 30 segundos se cancela la conexión
\end{lstlisting}

Número de intentos si falla el password 
\begin{lstlisting}
    R1(config)#ip ssh authentication-retries 3 //A los 3 intentos se cancela la conexión 
\end{lstlisting}

Configuración versión de ssh 
\begin{lstlisting}
    R1(config)#ip ssh version 2
\end{lstlisting}

Privilegios 
\begin{lstlisting}
    R1(config)#username admin privilege 15 pass 1234 //Asignamos un nivel de privilegios 15 con el password 1234
\end{lstlisting}

\begin{lstlisting}
    R1(config)#line vty 0 4 //Modo de configuración de una terminal virtual (remota)
    R1(config-line)#transport input ssh //Me comunicaré usando SSH
    R1(config-line)#login local //Desde dónde haré la verificación de password, imaginemos que tenemos un server que tiene los controles de acceso. En este caso todo se hará localmente
\end{lstlisting}

Verificamos usando R2 que todo esté bien 
\begin{lstlisting}
    R2#ssh -v 2 -l admin 10.10.0.1 //En mi caso no pude especificar la versión por tanto removemos ese comando
    R2#ssh -l admin 10.10.0.1 //Me pedirá el password el cual es 1234 y ahora puedo acceder remotamente a R1
\end{lstlisting}
Verificamos si esta activo el SSH 
\begin{lstlisting}
    R1(config)#end 
    R1show ssh
\end{lstlisting}

Niveles de privilegio 
\begin{lstlisting}
    R1conf t
    R1(config)#privilege line//Definimos el conjunto de comando a asignar
\end{lstlisting}

\section{Enrutamiento estático y dinámico}
Estas tienen una diferencia: Recordemos en el ejemplo anterior hacíamos ping a R2 pero no teníamos respuesta. En enrutamiento dinámico cada router sabe que va ha hacer, si envian una petición sabe por que interfaz enviar una respuesta.\\
Nosotros como adminsitradores definimos un registro en la tabla de enrutamiento.

\begin{lstlisting}
    R1show ip route //Vemos tablas de enrutamiento del router  
\end{lstlisting}

Si el paquete no está en alguno de los renglones de la tabla de enrutamiento no envia nada el router.       

\section{Protocolos CMIP}

\begin{itemize}
    \item {ACSE: Se encarga de establecer las conexiones entre gestor y agente.}
    \item {ROSE: Se encarga de generar la transferencia de datos entre gestor y agente.}
    \item {CMISE: Utiliza los protocolos anteriores para administrar la red, etc. Entre los agentes.}
\end{itemize}

\chapter{Enrutamiento estático y dinámico}

\section{Enrutamiento estático}
Es la forma más simple de enrutamiento, en la que las tareas de descubrimiento de rutas y su propagación en la red son realizadas manualmente por el administrador de la internetwork. (El administrador añade una dirección)

En un esquema de enrutamiento estático una vez que la ruta es configurada, no hay necesidad de que los routers intenten descubrimientos de rutas para comunicar entre si información acerca de las mismas.
\begin{itemize}
    \item {La distancia administativa predeterminada de una ruta estática es 1}
    \item {Las rutas estáticas serán incluidas en la tabla de ruteo a menos que la red esté dieractamente conectada, o que la interfaz de salida especificada no pueda ser alcanzada por el router}
\end{itemize}

Supongamos que tenemos configurados varios protocolos, el router le da prioridad a un protocolo sobre otro, esto es la distancia administrativa.

\section{Enrutamiento dinámico}
EN el enrutamiento dinámico, la determinación de una ruta se hace usando información que es obtenida de los protocolos de enrutamiento.

Esta información se genera en respuesta a cambios en la red.

\subsection{Ventajas del enrutamiento estático}
\begin{itemize}
    \item {Eficiencia de recursos: no consume ancho de banda para intercambio y descubrimiento de rutas, no ocupa procesamiento del CPU en cálculo de rutas y requiere menos memoria}
    \item {Mayor segurodad}
    \item {Control sobre el tráfico}
    \item {Respaldo sobre rutas dinámicas: puede configurarse una ruta estática con una distancia administrativa mayor a la ruta obtenida por un protocolo a un mismo destino, en caso de que el protocolo falle se añade una ruta estática.}
\end{itemize}

\subsection{Desventajas del enrutamiento estático}
\begin{itemize}
    \item {No se adapta a cambios a al topología}
    \item {Costo de administración}
    \item {No ofrece gran escalabilidad}
\end{itemize}

\subsection{Ventajas enrutamiento dinámico}
\begin{itemize}
    \item Redundancia
    \item Multiple acceso
    \item Carga compartida 
    \item Escalable y flexible
\end{itemize}

\subsection{Desventaja del enrutamiento dinámico }
\begin{itemize}
    \item Procesamiento
\end{itemize}

\subsection{ENrutamiento estático}
Se realiza utilizando ip route 
\begin{lstlisting}
    ip route prefijo máscara {dir-reencio | vlan vlan-id}[distancia][permanente][tag valor]
\end{lstlisting}

\begin{itemize}
    \item prefijo: identificador de red 
    \item máscara: máscara de la subred 
    \item dir-reenvio | vlan vlan-id: a dónde se dará un ''brinco''
    \item distancia (opcional): especifica la métrica de la distancia para esta ruta, los valores son válidos entre 1 y 255. Las rutas con valores pequeños son las preferidas
    \item Permanente (opcional). Especifica una ruta permanente
    \item tag (opcional): especifica una etiqueta para la ruta con valores válidos entre 1 a 4294967295
\end{itemize}

\section{Protocolos}
Hay dos tipos de protocolos, los protocolos enrutables y los de enrutamiento
\subsection{Protocolos enrutables}
Tienen una dirección que deja moverse.
\begin{enumerate}
    \item IP 
    \item IPX 
    \item AppleTalk
\end{enumerate}

\subsection{Protocolos no enrutables}
Funcianan sobre una sola LAN. 
\begin{enumerate}
    \item NetBEUI 
    \item DLC 
    \item LAT 
    \item DRP 
    \item MOP
\end{enumerate}

\subsection{Protocolo de enrutamiento }
Mecanismos necesarios para compartir la información de enrutamiento. Hace que los routers compartan información, para que construyan sus tablas de enrutamientO .
\begin{enumerate}
    \item RIP 
    \item OSPF 
    \item IGRP
\end{enumerate}

\subsubsection{Objetivos}
\begin{enumerate}
    \item Identificar rutas potenciales 
    \item Deteterminar rutas óptimas 
    \item Detectar cualquier cambio en las topologías
    \item Optimización 
    \item Simplicidad y bajo gasto
    \item Solidez y estabilidad 
    \item Flexibilidad 
    \item Convergencia rápida: Si existe un cambio en la topología hay un tiempo que se da para notificar a los routers estos cambios y recalcular los algoritmos con los que se construye la tabla de enrutambiento
\end{enumerate}


\begin{itemize}
    \item Proveen reglas sintácticas y semánticas
    \item Contienen los detalles del formato de los mensajes 
    \item Describen intercambio de mensajes 
    \item Describen los proceso de comunicación 
\end{itemize}

\section{Distancia administrativa }

Valores predeterminados

\begin{longtable}[c]{|
    >{\columncolor[HTML]{00009B}}l |l|}
    \hline
    {\color[HTML]{FFFFFF} Origen de la ruta}  & \cellcolor[HTML]{00009B}{\color[HTML]{FFFFFF} Valores predeterminados de la distancia} \\ \hline
    \endfirsthead
    %
    \endhead
    %
    {\color[HTML]{FFFFFF} Interfaz conectada} & {\color[HTML]{000000} 0}                                                               \\ \hline
    {\color[HTML]{FFFFFF} RUta estática}      & {\color[HTML]{000000} 1}                                                               \\ \hline
    {\color[HTML]{FFFFFF} Ruta resumen EIGRP} & {\color[HTML]{000000} 5}                                                               \\ \hline
    {\color[HTML]{FFFFFF} Protocolo BGP}      & {\color[HTML]{000000} 20}                                                              \\ \hline
    {\color[HTML]{FFFFFF} EIGRP interno}      & {\color[HTML]{000000} 90}                                                              \\ \hline
    {\color[HTML]{FFFFFF} IGRP}               & {\color[HTML]{000000} 100}                                                             \\ \hline
    {\color[HTML]{FFFFFF} OSPF}               & {\color[HTML]{000000} 110}                                                             \\ \hline
    {\color[HTML]{FFFFFF} IS-IS}              & {\color[HTML]{000000} 115}                                                             \\ \hline
    {\color[HTML]{FFFFFF} RIP}                & {\color[HTML]{000000} 120}                                                             \\ \hline
    {\color[HTML]{FFFFFF} EIGRP externo}      & {\color[HTML]{000000} 150}                                                             \\ \hline
    {\color[HTML]{FFFFFF} BGP interno}        & {\color[HTML]{000000} 200}                                                             \\ \hline
    {\color[HTML]{FFFFFF} Desconocido}        & {\color[HTML]{000000} 255}                                                             \\ \hline
\end{longtable}

\section{COncepto de métrica y sus componentes}
\begin{itemize}
    \item Un algoritmo de enrutamiento debe determinar las ventajas de una ruta sobre otra 
    \item Para cada ruta a través de la res, un protocolo utiliza valores llamados métricas para determinar cuál es laruta óptima 
    \item Las métricas pueden tomar como base una sola característica de la ruta, o pueden calcularse tomando en cuenta distancias características 
    \item Con las métricas, el algoritmo genera un valor llamado valor métrico
\end{itemize}

\subsection{Metrícas de enrutamiento}
\begin{enumerate}
    \item Ancho de banda
    \item Retardo 
    \item Carga: permite reajustar las rutas 
    \item Confiabilidad: índice de error 
    \item Número de saltos 
    \item Tictacs 
    \item Costo 
\end{enumerate}

\subsection{Clases de protocolos de enrutamiento}
\begin{itemize}
    \item Protocolos de enrutamiento Interior (IGPs)
    \item Protocolos de enrutamiento exterior (EGPS)
\end{itemize}

Cuando nos referimos a interno y externo hablamos de dominios 

EL método en que descubren y calculan rutas 
\begin{itemize}
    \item Protocolos vector-distancia 
    \item Protocolo de estado de enlace 
    \item Protocolos híbridos balanceados 
\end{itemize}

DIreccionamiento IP 
\begin{itemize}
    \item Protocolos classful 
    \item Protocolos classless
\end{itemize}

\subsubsection{Convergencia}
Si es menor el tiempo de convergencia es mejor. 

\section{Protocolos classful}
\begin{itemize}
    \item {Los protocolos de enrutamiento classfull no incluyen máscara de red en los anuncios de red}
    \item {Dentro de la misma red, se asume una consistencia de máscaras de subred}
\end{itemize}

\section{Protocolos classless}
Requerimos enviar las máscaras y construye una estructura jerárquica dependiendo de las direcciones IP y las máscaras que se estan utilizando 
\begin{itemize}
    \item RIPV2 
    \item OSPF
\end{itemize}

\section{Protocolos de enrutamiento interior}
Se utilizan para intercambias información dentro de un siste aautónomo.

\section{Principales tipos de enrutamiento}

\begin{itemize}
    \item {Vector-distancia: Determina la dirección a partir de pares, básicamente se tiene un router y ese router tiene el identificador de red y su costo (tabla de enrutamiento)}
    \item {Estado del enlace: }
\end{itemize}

\section{Ejemplo de configuración RIPv2}

\subsection{Configuración básica}
Configurar RIPv1
\begin{itemize}
    \item {Activar protocolo RIPv1 \\ router rip}
    \item {Anunciar redes: \\ network <dirección de red> }
\end{itemize}

Configurar RIPv2 en R4
\begin{itemize}
    \item {Activar el protocolo RIPv2: \\ router rip}
    \item {Especificar la versión 2: \\ version 2}
    \item {Anunciar redes: \\ network <dirección de red>}
\end{itemize}

\begin{lstlisting}
    Rx#write //Para mantener la configuración de los routers
\end{lstlisting}

Damos por hecho que las IP
\begin{lstlisting}
    R1# conf t
    R1(config)#router rip 
    R1(config-router)#version 2
\end{lstlisting}


Recordemos que RIP envia sus tablas de enrutamiento a los routers vecinos. Cada 30s se refresca esta información.   

\begin{lstlisting}
    R1# conf t
    R1(configure)# router rip 
    R1(configure-router)# version 2
    R1(configure-router)# network 200.1.1.0 //Subred de la computadora
    R1(configure-router)# network 40.0.0.0  //Subred con los demás routers
\end{lstlisting}

¿Porqué necesito saber de donde voy y a donde voy? Imaginemos que tenemos una topología con switches o algo intermedio entre routers, debemos saber hacia donde hay que moverse, por esa razón es que rip nos dice a donde vamos a ir.

Rip soporta balanceo de cargas hasta en cinco rutas.

Hay que indicarle a RIP que haga un superneteo de las redes ya que recordemos que RIP es orientado a clases:
\begin{lstlisting}
    R3# conf t
    R3(configure)#router rip
    R3(configure--router)#no auto-summary 
\end{lstlisting}

Más comandos para monitorizar 
\begin{lstlisting}
    R1#show ip protocols
\end{lstlisting}

Como evitamos sobrecarga en ciertas interfaces? Debemos indicarselo a rip 

\begin{lstlisting}
    R3#conf t
    R3(configure)#router rip 
    R3(configure-router)#passive-interface ethernet 0/0 //donde tenemos las LAN (las computadoras conectadas)
\end{lstlisting}

Supongamos que queremos configurar un router como router de frontera en nuestro caso el router R5, como lo configuramos?
\begin{lstlisting}
    R5#conf t
    R5(configure)#router rip 
    R5(configure-router)#default-information origin
\end{lstlisting}

Si hacemos show ip route en el router R3 tenemos una red por defecto la que es la 0.0.0.0 (último renglón)\\\\

¿Que pasa si quiero combinar un protocolo dinámico con uno dinámico? Para esto rip va a difundir información que no pertenece al protocolo RIP.

Requerimos informar al router Rf que toda subred que no esté en 200.20.20 y 60.6.6.4 salga por el router R5 ¿Cómo hacemos eso? Hay que definir primero un enrutamiento estático. Que ID de red necesitamos? 0.0.0.0 con la máscara 0.0.0.0 esto para capturar todas las IP que no caigan en las dos subredes que hemos mencionado antes. Nuestro siguiente punto de salto es 60.6.6.5
\begin{lstlisting}
    Rf(config)#ip route 0.0.0.0 0.0.0.0 60.6.6.5
\end{lstlisting}

Nótese que tenemos una prioridad de 1 en este caso. 

Debemos hacer lo mismo con el router R5 en este caso.

\begin{lstlisting}
    R5(config)#ip route 200.20.20.0 255.255.255.0 60.6.6.6
\end{lstlisting}

Los demás routers no conocen esta configuración así que este router debe de informar al resto de routers

\begin{lstlisting}
    R5(config)# router rip
    R5(config)# redistribute static
\end{lstlisting}

\section{Cuenta a infinito}
Cuando hablamos de RIP el número máximo de saltos o métrica es 15 si es 16 o mayor el destino es inalcanzable. Por ejemplo si removemos una conexión a una LAN y estamos usando RIP vamos a generar una especie de ciclo hasta que elk valor de los saltos llegue a 16 esto hace que el destino se convierta en inalcanzable.


\chapter{Protocolo OSPF}
\section{Aspecto básico del protocolo OSPF}
Definido por la IETF en la RFC 2328, de Abril-98.
\begin{itemize}
    \item {Se encapsula en IP con rpotocolo 59h}
    \item {Se define 05 tipos de mensajes}
    \item {Lectura obligada}
\end{itemize}

\section{Características}
\begin{itemize}
    \item {Se hace uso del protocolo HELLO}
    \item Se envía periodicamente a la dirección multicast IP 224.0.0.5
\end{itemize}

Requiere mucha memoria ya que genera una base de datos, requiere mucho ancho de banda, ya que envía la tabla de enrutamiento e información de la BD a los routers vecinos. EN este protocolo como ya se mencionó antes se utilizan 5 tipos de mensajes, dónde el más sencillo es el mensaje ''HELLO''. 
OSPF hace que los routers conozcan la información de toda la tabla de enrutamiento.

OSPF divide todo el sistema autónomo en áreas, dónde todas las áreas deben estar conectadas al área 0, dónde el área cero es el área de backbone es la columna vertebral de OSPF.
\subsection{Observaciones al estado de enlace}
\begin{itemize}
    \item {Los routers de enlace requieren más memoria y potencia de procsamiento, que un router vector-distancia}
    \item {Al inicio del proceso se debe inundar la red con mensaje LSA, puede degradar la red}
    \item {Hay que dividir la red en áreas}
\end{itemize}
\subsection{Configuración}
\begin{lstlisting}
    Rx#conf t
    Rx(config)#router ospf process-ID //Debe ser el mismo para todos los routers
    Rx(config-router)# network dirección_de_red wildcard /*comodín*/ area area_ID
    Rx(config-router)#exit
\end{lstlisting}
El wildcard es el inverso de la máscara por ejemplo si la máscara es 255.255.255.252 el inverso es 0.0.0.3

\section{Wildcard}
\subsection{Bits de máscara wildcard}
Una máscara wildcard es un número de 32 bits dividido en cuatro octetos (como IP)\\
El enmascaramiento wildcard es utilizado por las ACL para identificar una sola dirección o múltiples direcciones para pruebas d epermiso y denegación\\ 
No guardan relación funcional con las máscaras de subred\\
ALgunas veces, se refiere a una máscara de wildcard como una máscara inversa.\\\\

Un bit a 0 en la mmáscara significa comprobar el valor correspondiente y un 1 es ignorar el valor correspondiente.\\\\
Ejemplo: tenemos la dirección \\
30.15.25.0/26 supongamos que me interesan todas las direcciones IP pares de esta LAN. La máscara es 255.255.255.192\\\\

\begin{longtable}[H]{|l|l|l|l|}
    \hline
    \rowcolor[HTML]{FFFFFF} 
    {\color[HTML]{333333} 255}                  & {\color[HTML]{333333} 255}                  & {\color[HTML]{333333} 255}                 & {\color[HTML]{333333} 192}                      \\ \hline
    \endfirsthead
    %
    \endhead
    %
    \rowcolor[HTML]{FFFFFF} 
    {\color[HTML]{333333} 30}                   & {\color[HTML]{333333} 15}                   & {\color[HTML]{333333} 25}                  & {\color[HTML]{333333} 00xxxxxx}                 \\ \hline
    \rowcolor[HTML]{FFFFFF} 
    \multicolumn{4}{|l|}{\cellcolor[HTML]{FFFFFF}{\color[HTML]{333333} Los primeros 26 bits nos interesan es decir:}}                                                                        \\ \hline
    \rowcolor[HTML]{FFFFFF} 
    {\color[HTML]{333333} 0}                    & {\color[HTML]{333333} 0}                    & {\color[HTML]{333333} 0}                   & {\color[HTML]{333333} 00}                       \\ \hline
    \rowcolor[HTML]{FFFFFF} 
    \multicolumn{4}{|l|}{\cellcolor[HTML]{FFFFFF}{\color[HTML]{333333} El último bit nos define la paridad por tanto, solo el último bit nos importa saber que valor tiene (debe ser cero)}} \\ \hline
    0                                           & 0                                           & 0                                          & 0011111110                                      \\ \hline
    \multicolumn{4}{|l|}{Con lo cual obtenemos nuestro wild card}                                                                                                                            \\ \hline
    0                                           & 0                                           & 0                                          & 62                                              \\ \hline
\end{longtable}

Tomemos de nuevo el wildcard 0.0.0.3:
\begin{longtable}[H]{|
    >{\columncolor[HTML]{FFFFFF}}l |
    >{\columncolor[HTML]{FFFFFF}}l |
    >{\columncolor[HTML]{FFFFFF}}l |
    >{\columncolor[HTML]{FFFFFF}}l |
    >{\columncolor[HTML]{FFFFFF}}l |}
    \hline
    wildcard:                                  & {\color[HTML]{333333} 0}  & {\color[HTML]{333333} 0} & {\color[HTML]{333333} 0} & {\color[HTML]{333333} 00000011}                                 \\ \hline
    \endfirsthead
    %
    \endhead
    %
    IDred:                                     & {\color[HTML]{333333} 10} & {\color[HTML]{333333} 2} & {\color[HTML]{333333} 3} & {\color[HTML]{333333} 00000100}                                 \\ \hline
    \cellcolor[HTML]{FFFFFF}                   & {\color[HTML]{333333} 10} & {\color[HTML]{333333} 2} & {\color[HTML]{333333} 3} & {\color[HTML]{333333} 000001( \{00\}, \{01\}, \{10\}, \{11\} )} \\ \cline{2-5} 
    \cellcolor[HTML]{FFFFFF}                   & \multicolumn{4}{l|}{\cellcolor[HTML]{FFFFFF}{\color[HTML]{333333} Por tanto tenemos redes definidas desde}}                                       \\ \cline{2-5} 
    \cellcolor[HTML]{FFFFFF}                   & {\color[HTML]{333333} 10} & {\color[HTML]{333333} 2} & {\color[HTML]{333333} 3} & {\color[HTML]{333333} 4}                                        \\ \cline{2-5} 
    \cellcolor[HTML]{FFFFFF}                   & \multicolumn{4}{l|}{\cellcolor[HTML]{FFFFFF}\begin{tabular}[c]{@{}l@{}}.\\ .\\ .\end{tabular}}                                                    \\ \cline{2-5} 
    \multirow{-5}{*}{\cellcolor[HTML]{FFFFFF}} & 10                        & 2                        & 3                        & 7                                                               \\ \hline
\end{longtable}

\section{Aspectos de configuración}
\begin{itemize}
    \item {AL iniciarse el proceso OSPF en un router, el IOS utiliza la dirección IP activa local más alta como ID del router}
    \item {Si no existe una interfaz activa, el proceso OSPF no se iniciará}
    \item {Para asegurar la estabilidad del proceso OSPF, es necesario que el router tenga una interfaz activa en todo momento}
    \item {La interfaz loopback es importante para este objetivo}
\end{itemize}

Supongamos que tenemos las siguientes direcciones IP
\begin{itemize}
    \item 20.20.20.1 
    \item 20.20.30.1
    \item \textbf{20.20.40.1}
\end{itemize}
OSPF va a tomar la dirección IP \textbf{20.20.40.1} ya que es la más alta. Esto vuelve inestable todo OSPF ya que al conectar más redes no obtendremos a veces la dirección IP más alta, si la interfaz con la IP más alta falla nuevamente volvemos OSPF inestable. Es decir OSPF es dependiente de las interfaces con las que se está trabajando. Para solucionar esto utilizamos interfaces loopback.
Si se tienen varias ip de loopback nuevamente elige la que tenga la IP más alta.\\\\

Ahora debemos de responder lo siguiente ¿Cómo definimos el router designado y el designado de respaldo? Esto se realiza con la prioridad 
\begin{lstlisting}
    R(config)#interface serial 0/0/0
    R(config-if)#ip priority n_prioridad
\end{lstlisting}
Un valor de prioridad puede variar de 0 a 255 \\

Valor 0 de prioridad imposibilita al router que sea elegido DR \\

\chapter{Nivel de administración en OSI}
Una red tiene tres componentes básicos
\begin{itemize}
    \item Dispositivos administrados 
    \item Agentes 
    \item Sistemas administradores de la red
\end{itemize}
\section{ Estándares }
\begin{itemize}
    \item {CMIS: era la parte del protocolo encarga de dedefinir como se realiza la comunicación}
    \item {CMIP: como se comunicaban los elementos de la red }
\end{itemize}

\subsection{Elementos de CMIS}
\begin{itemize}
    \item {APlicación de sistemas SMAP (gestiona la red)}
    \item {Entidad de aplicación de gestión de sistemas (SMAE) esta del lado de los dispositivos se conecta con el SMAP para gestionar los dispositivos, esta gestión va desde la capa 1 hasta la capa 7.}
    \item {Entidad de gestión de nivel LME }
    \item {Base de información de gestión MIB }
\end{itemize}

\section{Manager information base MIB}
\subsection{Almacenamiento de la información}

\begin{itemize}
    \item {En un agente el almacenamiento de la información se realiza en objetos, los cuales están definidos por:
        \begin{enumerate}
            \item {Atributos}
            \item {Operaciones que realizan}
            \item {Notifiaciones que pueden emitir}
            \item {Interacción con otros objetos}
        \end{enumerate}
    }
\end{itemize}

La forma de representación y almacenamiento de los datos es un agente no se estandariza.
Funciones locales utilizadas para convertir la información relacionada con los objetos gestionados a un formato que se pueda almacenar localmente .
Una función específica se encarga de realizar las conversiones necesarias .
La información se almacena en objetos.

\subsection{MIB, estándar ISO 10165-1 (x 720)}
Define como se debe realizar la gestión de la información 

\subsection{Estructura de la MIB }
La unidad básica de información es el objeto
\begin{itemize}
    \item {Atributos}
    \item {Comportamientos}
    \item {Notificaciones}
\end{itemize}
Existen dos formas de crear jerarquias para estructurar la MIB 
\begin{itemize}
    \item {Definición de subclases, o clases derivadas de otras clases que heredan sus características}
    \item {Una instancia de objeto puede estar contenida en otra instancia de objeto}
\end{itemize}

\begin{longtable}[c]{|
>{\columncolor[HTML]{FFFFFF}}l |
>{\columncolor[HTML]{FFFFFF}}l |
>{\columncolor[HTML]{FFFFFF}}l |}
\hline
{\color[HTML]{333333} SMAP}   & \cellcolor[HTML]{FFFFFF}{\color[HTML]{333333} }                                                 & {\color[HTML]{333333} SMAE}   \\ \cline{1-1} \cline{3-3} 
{\color[HTML]{333333} Gestor} & \multirow{-2}{*}{\cellcolor[HTML]{FFFFFF}{\color[HTML]{333333} $\leftrightarrow$}} & {\color[HTML]{333333} Adente} \\ \hline
\endfirsthead
%
\endhead
%
\end{longtable}

MIB no está diseñado para la administración de redes si no la administración de empresas.

\chapter{Funciones de gestión de sistemas }
Son las distintas áreas que tenemos que considerar para el funcionamiento de nuestra red, tenemos 
\begin{itemize}
    \item Gestión de fallas 
    \item Gestión de costos
    \item Gestión de configuración: actualizaciones, documentación, etc.
    \item Gestión de prestaciones: que se le ofrece a los usuarios 
    \item Gestión de seguridad
\end{itemize}

Todos los gestores mencionados tienen relación con más elementos.\\ 

\section{Áreas funcionales }
Se les conoce como FCAPS 
\subsection{Fallas}
Busca encontrar dónde ocurren los errores pero además de solucionarlos busca mecanismos para que las soluciones ocurran lo más rápido posible 
\subsection{COnfiguración}
BUsca una automatización de confiuraciones de red. 
\subsection{Contabilidad}
Uso de recursos de nuestros clientes 
\subsection{Desempeño}
Que tan eficiente es nuestra red, usa estadísticas, alarmas, etc.
\subsection{Seguridad}
Todo está conectado donde debe estar, se conecta quien debe ser, etc. 

\section{Acciones básicas }
Lectura(sondeo), Justificacion(trampa), escritura

\chapter{Gestión de fallas}
Tiene como objetivo registrar, detectar y contrarestar las condiciones de fallo de una red.

\section{¿Qué implica administrar fallas?}
\begin{itemize}
    \item Identificación del prolblema
    \item Crear, aprobar la solución y se propaga 
    \item Resolución del problema y documentación
\end{itemize}


\chapter{EIGRP}
Es una versión de IGRP mejorada por CISCO. 

\section{Objetivos}
\begin{itemize}
    \item ¿Como está configurado? 
    \item Historia 
    \item Comandos báiscos 
    \item Calcular la métrica compuesta por EIGRP 
    \item Describir los conceptos y el funcionamiento de DUAL 
    \item Describir los usos de los comoandos adicionales de EIGRP
\end{itemize}

Tiene una característica importante, nos permite trabajar de distintas formas, permite trabajar con el protocolo DUAL o vector distancia o bien estado de enlace. También permite configurar las prioridades dependiendo de la fuente de información. 

En la capa de transporte usa el protocolo RTP. Tiene ciertas catracterísticas especiales 
\begin{itemize}
    \item {Soporta envios confiables y no confiables dependiendo de las necesidades}
    \item {Soporte para unicast o multicast}
\end{itemize}

También tiene 5 tipos de paquetes 
\begin{itemize}
    \item Saludo: permite detectar routers vecinos para establecer adyacencias e informar que estamos activos
    \item Actualización 
    \item Reconocimiento (En respuesta a una actualizacion)
    \item Consulta 
    \item Respuesta a la consulta
\end{itemize}

\section{Tiempo en hold }
Tiempo en espera que un router maneja antes de recibir un mensaje de saludo. Si en cierto tiempo no se recibe un mensaje de saludo se entiende como que el router tiene problemas y lo marca. Si se rebasa un segundo tiempo de espera se da por hecho que el router está inactivo y se elige otro router por el cual pasar.    


\chapter{Implementación de VLAN's y troncales}
Al tener un primer diseño de una red es común no pensar que nuestra red puede crecer, debemos considerar varias cosas al realizar el diseño de una red. 
Aquí tenemos una pequeña lista de errores comunes en las redes.
\begin{itemize}
    \item Dominios de falla; Es más probable que ocurra alguna situación si perdemos una sección perdemos varias LANS 
    \item Dominios de broadcast
    \item Gran cantidad de trfico unicast con MACs desconocidas
    \item Trafico multicast en puertos donde no se requiere 
    \item Dificultad en el manejo y soporte
    \item Posibles vulnerabilidades de seguridad
\end{itemize}

\section{Agrupando funciones del negocio dentro de VLANs}
Las VLANs nos permiten una separación lógica de las LAN aún que estas no se encuentren físicamente contiguas. 

\begin{itemize}
    \item {Direccionamiento jerárquico de red significa que un número de red es asignado a una VLAN}
    \item {Beneficios: 
        \begin{enumerate}
            \item {Fácil mantenimiento y resolución de}
            \item {Errores minimizados}
            \item {Tablas de enrutamiento reducumas}
        \end{enumerate}}
\end{itemize}

\subsection{Describiendo tecnologías de interconexión}
Cuando estemos en secciones donde hay una concentración de envíos de paquetes debemos utilizar tecnologías de mayor ancho de banda, a continuación se muestra un listado de tecnologías que son posibles utilizar, desde la que debe ser utilizada desde abajo hacía arriba.
\begin{itemize}
    \item Fast ethernet
    \item Gigabit ethernet
    \item 10-gigabit ethernet
    \item EtherChannel
\end{itemize}

\subsection{Determinando el equipo y el cableado a necesitar}
Los cuatro objetivos en el diseño de una red de alto desempeño son
\begin{itemize}
    \item Seguridad: la información de un departamento no sale d euna sección 
    \item Disponibilidad 
    \item Escalabilidad 
    \item Manejabilidad
\end{itemize}

\section{Enlace troncal}
Es un enlace de nuestra red donde circulan tramas de distintas VLANs.

\begin{itemize}
    \item {Un enlace troncal se puede comprarar con las carreteras d euna autopista}
    \item {Las carreteras que tienen distintos puntos d einicio y fin que comparten una autopista principal durante algunos kilimetros, luego se vuelven a dividir par allegar a destinos diferentes}
    \item {Este método es considerablemente económico}
\end{itemize}

EN resumen es un enlace punto a punto que soporta varias VLANs y nos permite el ahorro entre distintos puertos.

\subsection{Explicando enlaces troncales}

Hay dos tecnologías principales 
\begin{itemize}
    \item {ISL (CISCO)
        \begin{itemize}
            \item Soporta multiples protocoloas de capa 2 (Ethernet, Token Ring, FDDI y ATM)
            \item Soporta PVST 
            \item No usa VLAN nativa, solo encapsula cda trama 
            \item El proceso de encapsulación deja las tramas originales sin modificación
        \end{itemize}}
    \item {802.1Q (IEEEE standard trunking protocol)
        \begin{itemize}
            \item Permite la transmisión en un enlace físico troncal 
            \item Permite ethernet y token ring 
            \item Sporta hasta 5096 VLANs
            \item Soporta STP, RSTP 
            \item Soporta topologías punto a multipunto 
            \item Facilita el tráfico 
            \item Soporta QoS (Quality of service )
        \end{itemize}}
\end{itemize}

Estos dos protocolos son necesarios para cuando se necesite interconectar 
\begin{itemize}
    \item Dos switch 
    \item Switch y router 
    \item Switch y tarjeta nick
\end{itemize}
Asignamos el modo trunk a un puerto
\begin{lstlisting}
    Switch(config)#interface [Interfaz]
    Switch(config-if)#switchport mode trunk    
\end{lstlisting}

Configuración del router para VLAN 
\begin{lstlisting}
    Router>en 
    Router#configure t 
    Router(config)#interface fastEthernet 0/0
    Router(config-if)#no shutdown 
    Router(config-if)#interface fastEthernet 0/0.1 
    Router(config-subif)#encapsulation dot1Q [Número de la VLAN]
    Router(config-subif)#ip address [IP] [Mascara]
\end{lstlisting}

\section{Rangos de VLANs}
Cada VLAN en la red debe tener un VID único  \break 

El rango válido configurable por el usuario es 
\begin{itemize}
    \item VLAN ISL de 1 a 1024
    \item -----
\end{itemize}

Hay algunas VLANs reservadas

\begin{longtable}[c]{|l|l|l|l|}
    \hline
    \rowcolor[HTML]{00009B} 
    {\color[HTML]{FFFFFF} VLAN ranges} & {\color[HTML]{FFFFFF} Range}     & {\color[HTML]{FFFFFF} Use}                                                                                         & {\color[HTML]{FFFFFF} VTP propagated} \\ \hline
    \endfirsthead
    %
    \endhead
    %
    \rowcolor[HTML]{FFFFFF} 
    {\color[HTML]{333333} 0, 4096}     & {\color[HTML]{333333} reservado} & {\color[HTML]{333333} Uso del sistema. VLANs no vistas o usadas}                                                   & -                                     \\ \hline
    \rowcolor[HTML]{FFFFFF} 
    1                                  & Normal                           & \begin{tabular}[c]{@{}l@{}}VLAN por defecto. Esta VLAN puede ser\\ usada pero no borrada o modificada\end{tabular} & Si                                    \\ \hline
    \rowcolor[HTML]{FFFFFF} 
    2-1001                             & Normal                           & Pueden ser creadas, usadas y borradas                                                                              & Si                                    \\ \hline
    \rowcolor[HTML]{FFFFFF} 
    1002-1005                          & Normal                           & \begin{tabular}[c]{@{}l@{}}VLANs por defecto para FDDI y \\ Token Ring estas no pueden ser borradas\end{tabular}   & Si                                    \\ \hline
    \rowcolor[HTML]{FFFFFF} 
    1006-4094                          & Extendido                        & Solo para VLANs Ethernet                                                                                           & No                                    \\ \hline
\end{longtable}

\section{VTP}
Es un protocolo de capa 2. Mantiene las configuraciones de la VLAN consistentes, permite borrar, cambiar, agregar los nombres de la VLAN en todos los switches del dominio VTP\break

Características
\begin{itemize}
    \item Protocolo de cisco 
    \item Anuncia VLANs de la 1 a la 1005 solamente 
    \item Actualziaciones e intercambio através de solo enlaces troncales 
\end{itemize}

Permite compatibilidad y hay versiones de 1 a 5\break 

\section{Modos de VTP}
\subsection{Servidor}
\begin{itemize}
    \item Crea, modifica y borra VLANs
    \item Manda y envía avisos
    \item Sincroniza configuraciones de VLANs 
    \item Graba la configuración en NVRAM
\end{itemize}

\subsection{Client}
\begin{itemize}
    \item No puede crear, cambiar o borrar VLANs
    \item Re-envia los avisos 
    \item Sincroniza configuraciones de VLANs 
    \item No graba a NVRAM
\end{itemize}

\subsection{Transparent}
\begin{itemize}
    \item Crea, modifica y borra VLANS 
    \item Re-envía avisos 
    \item No sincroniza configuraciones de VLANs
    \item Graba la configuración a NVRAM
\end{itemize}

\section{Describiendo la operación de VTP}
\begin{itemize}
    \item {Administrador crea una VLAN}
    \item {Revisión 3 se actualiza a versión 4}
    \item {VTP propaga revisión 4}
    \item {Revisión 3 actualiza a 4}
    \item {VRP sincroniza y re-envía la información}
\end{itemize}

\section{VTP pruning}

\begin{itemize}
    \item VTP pruning usa los avisos de VLAN para determinar cuando una conexión troncal está mandando tráfico innecesario 
    \item VTP pruning incrementa el ancho de banda restringindo el flujo de tráfico en enlaces troncales dónde no se necesita
    \item {Restricciones:
        \begin{enumerate}
            \item En la VLAN 1 no se puede activar esta opción 
            \item Sólo se puede implementar VTP pruning solo en los servidores de VTP
        \end{enumerate}}
\end{itemize}


\section{Actividad}

\begin{lstlisting}
    SW1(config)#interface 0/0
    SW1(config-if)#switchport acc
    SW1(config-if)#switchport access vlan 20
    SW1(config-if)#exit
    SW1(config)#end
\end{lstlisting}

Vamos a configurar dos VLANS con los siguientes datos 
\begin{itemize}
    \item {
        \begin{itemize}
            \item {IDRED 120.120.20.129/25}
            \item {Rango válido: .129 - .190}
            \item {Broadcast: .191}
        \end{itemize}
    }
    \item {
        \begin{itemize}
            \item {IDRED 120.120.20.124/25}
            \item {Rango válido: .193 - .254}
            \item {Broadcast: .255}
        \end{itemize}
    }
\end{itemize}

En la máquina tenemos \break
IP: 120.120.20.194 \break
Máscara 255.255.255.192 \break
Default gateway: 120.120.20.193 \break

En nuestro diagrama la subred izquierda tiene la dirección \break 
120.120.128/26 \break 

La subred derecha \break
120.120.128/26 \break 

En la VM movemos la dirección a 120.120.20.130 con máscara 26 y gateway 120.120.120.129

Añadimos un router nuevo y vamos a configurar nuestro switch 

\begin{lstlisting}
    SW1# conf t
    SW1(config)#interface vlan 20 
    SW1(config-if)#ip address 120.120.20.129 255.255.255.192
    SW1(config-if)#no shutdown
    SW1(config-if)#exit
    SW1(config)#interface vlan 30
    SW1(config-if)#ip address 120.120.20.193 255.255.255.192
    SW1(config-if)#no shutdown 
    SW1(config-if)#exit 
    SW1(config)#interface fastEthernet 0/15 
    SW1(config-if)#switch mode trunk //Configuramos como modo troncal
    SW1(config-if)#exit 
    SW1(config)#end 
    SW1#
    SW1#show ip interface brief //Podemos ver nuestras VLAN
\end{lstlisting}

Ahora debemos configurar nuestro router 

\begin{lstlisting}
    R1#conf t
    R1(config)#interface ethernet 0/0
    R1(config-if)#no shutdown 
    R1(config-if)#interface ethernet 0/0.20 //Subinterface para las vlan 
    R1(config-subif)#encapsulation dot1Q 20 //Tipo de protocolo para vlan con el que estamos trabajando
    R1(config-subif)#ip address 120.120.20.129 255.255.255.192
    R1(config-subif)#exit 
    R1(config)#interface ethernet 0/0
    R1(config-if)#interface ethernet 0/0.30 
    R1(config-subif)#encapsulation dot1Q 30 
    R1(config-subif)#ip address 120.120.120.193 255.255.255.192
    R1(config-subif)#exit
    R1(config)#exit 
\end{lstlisting}

Ahora añadimos otro switch a la mísma topología. 

Switch 1 levantamos VTP
\begin{lstlisting}
    SW1#vlan database 
    SW1(vlan)#vtp password
    SW1(vlan)#vrp domain 4CM6
    SW1(vlan)#vtp pruning
    SW1(vlan)#vtp server
    SW1(vlan)#apply
    SW1(vlan)#exit 
    SW1#show vtp status
\end{lstlisting}
Ahora configuramos el nuestro switch
\begin{lstlisting}
    SW2#vlan database
    SW2(vlan)#vtp server
    SW2(vlan)#apply 
    SW2(vlan)#exit 
    SW2#
\end{lstlisting}

\begin{lstlisting}
    SW1#vlan database
    SW1(vlan)#vtp domain 4CM5
    SW1(vlan)#vtp password 1234 
    SW1(vlan)#exit  
    SW1#vtp switch
\end{lstlisting}    

\begin{lstlisting}
    SW1#conf t
    SW1(cofig)#interface fa0/4
    SW1(cofig-if)#switchport mode trunk 
    SW1(cofig-if)#no shutdown  
    SW1(cofig-if)#exit 
    SW1(cofig)#end
    SW1#vlan database 
    SW1(vlan)#vlan 40 name PRUEBA40
    SW1(vlan)#save 
    SW1(vlan)#exit
\end{lstlisting}

\begin{lstlisting}
    SW3#conf t
    SW3(cofig)#interface fa0/4
    SW3(cofig-if)#switchport mode trunk 
    SW3(cofig-if)#no shutdown  
    SW3(cofig-if)#exit 
    SW3(cofig)#end
    SW3#vtp domain 4CM4
    SW3(vlan)#vtp password 1234 
    SW3(vlan)#vtp client 
    SW3(vlan)#exit 
    SW3#vtp domain  
\end{lstlisting}

\chapter{Etherchannel}
Es una tecnología para aprovechar los enlaces o interfaces que tenemos libres.

Permite la agrupación lógica de varias conexiones físicas y permite manegar esta como una sola conexión. Soporta un máximo de 8 puertos FastEthernet, GigaEthernet o 10GigaEthernet. Es bastante robusta y estable. Es escalable además se puede usar en cualquier pare de la red y también permite mejorar el rendimiento. \break 

\section{Compatibilidad}
\begin{itemize}
    \item VLAN trunk 
    \item VTP 
    \item IEE 802.3Q 
    \item CISCO ISL 
    \item STP
\end{itemize}

El ether channel debe estar en el mismo switch, permite también trabajar con fibra óptica.

Podemos aplicar ether channel con dos protocolos PAGP y LACP

\subsection{PAgp, Port aggregation Protocol}
Propiedad de cisco.  Si se hace una configuración de enlace se debe hacer no sobre la interfaz sino el canal.

\begin{itemize}
    \item Describe, permite que el puerto negocie el establecimiento EtherChannel mediante PAgP 
    \item Auto, espera a recibir paquetes para negociar EtherChannel 
    \item Off, evita que los puertos establezcan un EtherChannel 
    \item On, activa el EtherChannel pero deshabilita PAgP
\end{itemize}

\subsubsection{LAPC, LInk Aggregation Control Protocol}
Propiedad de IEEE 802.3ad. Este soport ahasta 16 puertos ether channel. Tenemos los musmos modos que el anterior pero con diferentes nombres, Active, Passive, Off y On.

\section{Configuración}
Para PAgP
\begin{lstlisting}
    S1#conf t
    S1(config)#interface range gigabitethernet 0/1 - 4
    S1(config-if-range)#channel-protocol pagp 
    S1(config-if-range)#channel-group 1 mode desirable
    S1(config-if-range)#exit 
    S1(config)#exit 
\end{lstlisting}

Para LACP
\begin{lstlisting}
    S1#conf t
    S1(config)#interface range gigabitethernet 0/1 - 4
    S1(config-if-range)#channel-protocol lacp 
    S1(config-if-range)#channel-group 1 mode active
    S1(config-if-range)#exit 
    S1(config)#exit 
\end{lstlisting}

General estos modos se usan como puertos troncales 

\begin{lstlisting}
    S1#conf t
    S1(config)#interface range gigabitethernet 0/1 - 4
    S1(config-if-range)#channel-protocol pagp 
    S1(config-if-range)#channel-group 1 mode on
    S1(config-if-range)#exit 
    S1(config)#interface port-channel 1
    S1(config-if)#switch port mode trunk 
    S1(config-if)#switchport trunk allowed vlan 1-2
    S1(config-if)#exit 
    S1(config)#exit
\end{lstlisting}

\chapter{Protocolo spanning tree}
Nos evita redundancias(loops) a nivel de capa 2. STP se encarga de evitar esto, lo hace mediante bloqueos.

También a esto se le conoce como toermenta de broadcast. Es un evento que no es deseable. 