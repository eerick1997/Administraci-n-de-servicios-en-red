\chapter{Unidad II}
\section{Modelo OSI}
Pensemos en administración en la integración de varios elementos para manejar de forma eficiente y eficaz el manejar las edes de computadoras.

\begin{itemize}
    \item {Administrar: planificación de la mejor manera para una mejor gestión de la red-}
    \item {Gestionar: implementación de modificaciones y crreccionaes para alcanzar los objetivos de la red}
\end{itemize}

\subsubsection{Administración de sistemas}
Suma total de las políticas y procedimientos que intervienen en la configuración, control y monitoreo que conforman una red, con el fin de aseurar el eficiente y efectivo empleo de sus recursos. \\\\

Se le solicita a la ISO que diseñe un modelo de administración OSI 

\subsubsection{Modelo de administración OSI}

La Organización de Estándares INternacionales creó una comisión para crear un modelo de administración de redes, bajo la dirección del grupo OSI.

Surge como un modelo que involucra tanto la PC como la red, buscando una coordinación e integración entre si aún que se traten de modleos distintos.

