\chapter{Unidad II}
\section{Modelo OSI}
Pensemos en administración en la integración de varios elementos para manejar de forma eficiente y eficaz el manejar las edes de computadoras.

\begin{itemize}
    \item {Administrar: planificación de la mejor manera para una mejor gestión de la red-}
    \item {Gestionar: implementación de modificaciones y crreccionaes para alcanzar los objetivos de la red}
\end{itemize}

\subsubsection{Administración de sistemas}
Suma total de las políticas y procedimientos que intervienen en la configuración, control y monitoreo que conforman una red, con el fin de aseurar el eficiente y efectivo empleo de sus recursos. \\\\

Se le solicita a la ISO que diseñe un modelo de administración OSI 

\subsubsection{Modelo de administración OSI}

La Organización de Estándares INternacionales creó una comisión para crear un modelo de administración de redes, bajo la dirección del grupo OSI.

Surge como un modelo que involucra tanto la PC como la red, buscando una coordinación e integración entre si aún que se traten de modleos distintos.

\subsubsection{Modelo de administración de redes OSI (OSI-NMM)}
Es un modelo estándar que proporciona el marco conceptual para la organización de una amplia gama de recursos de la red.\\\\
Planificación de la capacidad de red\\\\
Gestión del rendimiento de la red\\

Comprende la administración de sistemas que delimita la operación de cualquiera de las 7 capas del modelo OSI, y la administración de los objetos gestionados,
Plantea los modelos de:
\begin{itemize}
    \item Organización
    \item Información
    \item Comunicación
    \item Función
\end{itemize}

\subsubsection{Modelo organizacional}
Describe lso componentes de la administración de redes tales como administrador, agente y otros, y sus interrelaciones.
Sus relaciones vienen dadas por la arquitectura de red.\\\\
El modelo organizacional del modelo OSI define los bloques y la relación entre estos.\\\\

Es una estructura dividida en dominios de red, los cuales comprenden su operabilidad y ofrece soporte de los aspectos de gestión del mismo.\\\\
Define conceptos para una gestión cooperativa, como para una gestión basada en jerarquías:
\begin{itemize}
    \item COncepto simétrico - entre dominios
    \item Concepto asimétrico - entre dominios y subdominios
\end{itemize}

\paragraph{Dominio ejemplo}
www.ipn.mx \\
\textbf{www.escom.ipn.mx}\\
\textbf{www.saes.escom.ipn.mx}
Dónde el tercer y segundo ejemplo son subdiminios del primero 
\subsubsection{Gestión de dominios}
Define la división de entorno, teniendo en cuenta dos moibos principales:
\begin{itemize}
    \item {Políticas funcionales, donde se incluyen políticas de seguridad contabilidad etc}
    \item {Políticas no funcionales, como la gestión geográfica, tecnologías, etc}
\end{itemize}
\subsubsection{Sub modelo informativo}
Trata de la estructura y almacenamiento de la información relativa a la administración de la red \\\\
Esta información se guarda en una base de datos la cual recube nombre de base de datos de información de administración \textbf{(MIB)} Es un archivo que guarda información sobre nuestras redes, el cual se encuentra en todas las capas de administración.
\subsubsection{Sub modelo conunicacional}
Habla de la forma como se comunican los datos de administración en el proceso gestir-agente\\\\
Atiende lo relacionado con el protocolo de transporte, el protocolo de aplicaciones y los comandos y respuestas entre pares. (Como me comunico, como lo hago, formato de mis tramas, que protocolos usaré, etc.)

\subsubsection{Modelo funcional}
Divide la complejidad de la administración en áreas funcionales de administración e intenta especificar funciones de administración genéricas.\\\\
EL modelo funcional proporciona las bases para construir librerías y soluciones.\\\\
áreas de administración del modelo OSI
\begin{itemize}
    \item {Administración de fallas (fault management)}
    \item {Administracion de configuración (configuration management)}
    \item {Administración de estadísticas y contabilidad (accounting amanager) [Comportamiento de la red y si todo está dentro de los rangos permitidos, además de saber cuanto se va a cobrar]}
    \item {Administración de desempeño (performance management) [Que todo este funcionando bien, saber si es posible hacer mejoras, etc.]}
    \item {Administración de seguridad (security management) [Quién tiene acceso a que si quien trata de accesar tiene permisos, etc]}
\end{itemize}

\subsection{CMIS (Servicios de interopabilidad de gestión de contenidos)}
Es un estándar, solo sabemos que cosas debe hacer. CMIS permite la interoperabilidad entre los distintos servicios que tenemos, el gestor hace pull-in y recaba información de todos los agentes.

\textbf{T-1.1 Levantar la siguiente topología GNS3 192.168.0.0/24: Bajar las ISOS router3600, Switch3600 crear dos máquinas virtuales basadas en linux. }
\begin{lstlisting}
    enable
    conf 
    configure ter 
    configure terminal 
    inte
    interface e 
    ip add 
    ip address
    ip address 192.268.0.1 255.255.255.128
    no shutdown
\end{lstlisting}