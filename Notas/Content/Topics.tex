\chapter{Introducción}
\section{Temario}
\begin{enumerate}
    \item {Administración de redes de computadoras
        \begin{enumerate}
        \item{Administración de redes de computadoras}
        \item{Administración de redes en el modelo OSI}
        \item{Servicios de administración común de información (CM15)}
        \item{Protocolo de administración comín de información (CM1P)}
        \item{El nivel de administración en OSI}
        \item{Administración del sistema (SMAP, SMAE, SMASE)}
        \item{Administración de fallas}
        \item{Administracion de configuraciones}
        \item{Administracion de rendimiento}
        \item{Administracion de seguridad}
        \item{Administracion de objetos}
        \item{Monitoreo de la carga de trabajo}
    \end{enumerate}
    }
    \item{ Administración de switch y ruteadores
        \begin{enumerate}
            \item{ Configuración básica de switch y ruteadores
                \begin{enumerate}
                    \item{Asignación de nombres y contraseñas}
                    \item{Configuración de interfaces}
                    \item{Copias de respaldo}
                    \item{Ruteo estático}
                    \item{Ruteo dinámico (RIP, OSPF, IGRP)}
                    \item{Administración del tráfico IP}
                    \item{Traducción de direcciones de red}
                    \item{Redes de área local virtual (VLANs)}
                \end{enumerate}
            }
        \end{enumerate}
    }
    \item { Protocolo simple de administración de red (SNMP)
        \begin{enumerate}
            \item {Introducción a SNMP}
            \item {Administraciones de alarmas SNMP}
            \item {Bases de datos de administración MIB}
            \item {Tipos y estructuras de paquetes SNMP}
            \item {SNMPv3}
            \item {Capas de comunicación}
            \item {Ventajs y desventajas de la implantación de un administrador SNMP}
        \end{enumerate}
    }
    \item { Monitorización de la administración de red
        \begin{enumerate}
            \item {El proceso y principios de monitorización}
            \item {Monitorizción para la administración de redes}
            \item {Recolección, análisis y notificación}
            \item {Análisis de tráfico y su limitación}
            \item {Los sitemas NSM}
            \item {Arquitectura de RMON}
            \item {RMON}
            \item {Comparación de RMON y RMON 2}
        \end{enumerate}
    }

    \item{ Calidad de servicio en red
        \begin{enumerate}
            \item {Introducción}
            \item {Calidad de servicio en internet}
            \item {Servicios integrados}
            \item {Protocolo RSVP}
            \item {Arquitectura de servicios diferenciados}
            \item {MPLS (MultoProtocol label switching)}
        \end{enumerate}
    }

    \item{ Administración del sistema
        \begin{enumerate}
            \item {Configuración y servicios de red}
            \item {Convivencia de los sitemas operativos}
            \item {Servicores DNS y DHCP}
            \item {Servidores de correo electrónico y POP}
            \item {Servidores de red}
            \item {Entornos PXE}
        \end{enumerate}
    }
\end{enumerate}

\begin{comment}
    Emulador: es lo más parecido a una red real
    Eulador a utilizar: GNS3
\end{comment}

\section{Evaluación}
\begin{table}[H]
    \centering
    \begin{tabular}{|
    >{\columncolor[HTML]{00009B}}l |l|l|l|}
    \hline
    {\color[HTML]{FFFFFF} }                & \cellcolor[HTML]{00009B}{\color[HTML]{FFFFFF} Primer parcial} & \cellcolor[HTML]{00009B}{\color[HTML]{FFFFFF} Segundo parcial} & \cellcolor[HTML]{00009B}{\color[HTML]{FFFFFF} Tercer parcial} \\ \hline
    {\color[HTML]{FFFFFF} Prácticas}       & 40\%                                                          & 40\%                                                           & 40\%                                                          \\ \hline
    {\color[HTML]{FFFFFF} Examen teórico}  & 30\%                                                          &                                                                &                                                               \\ \hline
    {\color[HTML]{FFFFFF} Examen práctico} & 30\%                                                          & 60\%                                                           &                                                               \\ \hline
    {\color[HTML]{FFFFFF} Proyecto final}  &                                                               &                                                                & 60\%                                                          \\ \hline
    {\color[HTML]{FFFFFF} Tareas}          & +10\%                                                         & +10\%                                                          & +10\%                                                         \\ \hline
    \end{tabular}
\end{table}

Miércoles 8:30 - 12:00, edificio central, al lado del laboratorio de física

\section{Fechas importantes}
\begin{itemize}
    \item {6 de septiembre, examen primer parcial práctico y teórico entrega 13 de septiembre }
    \item {18 de octubre, examen segundo parcial, entrega 25 de octubre. }
    \item {28 de octubre, proyecto final, entrega 29 de noviembre. }
\end{itemize}

\section{Reglas}
\begin{itemize}
    \item {Tareas y prácticas 1 semana para ser subida al moodle}
    \item {Semana extra para entregar (calificación sobre 5)}
\end{itemize}

\section{Bibliografía}
\begin{itemize}
    \item {Henshall, Shaw S. (1990). OSI Explained, End-to-end Computer Communication Standards. 2nd ed. England, Ellis Horwood Ed.}
    \item {Lewis, C. (1999) Cisco switched Internetworks. VLANs, ATM, Voice/Data Integration 1st Ed. Editorial, McGraw Hill}
    \item {Stalling, W. (2004) Redes e internet de alta velocidad, rendimiento y calidad de servicio}
    \item {Stalling, W. (1999) SNMP, SNMPv2, SMNPv3 and RMON y RMON2, Ed. Addison-Wesley}
    \item {Alegria I. Cortiñas R. (2005) Administración del sistema y la red, LINUX Editorial Person}
\end{itemize}